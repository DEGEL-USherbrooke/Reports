\section{Organisation du projet}
	\subsection{Mission et objectifs}
	Pour le projet de S6, l’équipe a pour mission de créer une application mobile afin de permettre aux étudiants d’accéder aux services du site web GEL via un téléphone mobile. Ce projet vise à apporter de la mobilité et de l’accessibilité à l’utilisateur. De plus, cette application permettra au département de génie électrique et de génie informatique de s’adapter aux nouvelles technologies tout en réduisant le taux d’absentéisme en cas de changement d’horaire. Ultimement, celle-ci améliorera la communication entre le département et ses étudiants en réduisant les délais et en s’assurant que tous les éléments importants à communiquer seront transmis efficacement.

	Plusieurs objectifs sont visés en ce qui concerne ce projet. Tout d’abord, l’utilisateur utilisera une application mobile lui permettant de se connecter qu’une seule fois par trimestre et aura accès à la dernière version de son horaire. Celui-ci aura aussi accès à un service de notifications avec plusieurs options afin d’être au courant de tout changement dans l’horaire. L’utilisateur configurera ses préférences pour recevoir ses notifications. Il pourra être notifié par l’application mobile, par courriel et/ou par messagerie texte. Un second objectif sera d’offrir le même service pour la grille de notes d'évaluation. Advenant un déroulement rapide du projet, l’équipe offrira un service identique s’appliquant aux messages écrits par les professeurs.
	
	\subsection{Intrants}
	Plusieurs intrants sont prévus pour l’application mobile. Il y a bien entendu l’utilisateur de l’application mobile, pour lequel il faudra développer une interface utilisateur simple, efficace et solide qui saura réagir face à tout comportement imprévu. Un autre intrant nécessaire est le serveur CAS utilisé par l’université de Sherbrooke, lequel permet à un étudiant de s’authentifier. Ensuite, comme l’équipe souhaite offrir à l’utilisateur un accès à son horaire le plus récent, elle s’est entendue avec M.~Bernard Beaulieu pour avoir la base de données incluant les horaires des étudiants comme intrant à son projet. Finalement, une autre équipe a décidé de concevoir un service de notifications généralisées. Alors, notre équipe a décidé d’utiliser leur service comme intrant au projet. Toutefois, le succès de notre projet dépendra pas directement de la réussite de leur service puisqu’une gestion des notifications sera présente directement dans l’application.

	\subsection{Échéancier préliminaire}
	Le projet se déroulera en 13 semaines, organisées en 3 sprints suivant la méthode agile SCRUM.
	\begin{table}[hp]
		\centering
		\caption{Échéancier préliminaire}
		\renewcommand{\arraystretch}{2}
\begin{tabular}{lll}
\hline
\bf Semaine & \bf Date & \bf Tâches \\
\hline
\hline
1 &
03/05 - 06/05 &
Organisation préliminaire du projet \\
%
2 &
07/05 - 13/05 &
Définition du projet - Création du MAP \\
%
\multirow{2}{*}{3} &
\multirow{2}{*}{21/05 - 27/05} &
23/05 : Remise du MAP \\
\rowcolor{green}
&& Début de la première itération \\
%
\rowcolor{green}
4 &
28/05 - 03/06 &
31/05 : Retour sur le MAP \\
%
\rowcolor{green}
\multirow{2}{*}{5} &
\multirow{2}{*}{04/06 - 10/06} &
07/06 : Fin de la première itération - Démo - Rapport d’avancement \\
&& \cellcolor{green!75!black} Début de la deuxième itération \\
%
\rowcolor{green!75!black}
6 &
11/06 - 17/06 & \\
%
\rowcolor{green!75!black}
7 &
18/06 - 24/06 &
21/06 : Fin de la deuxième itération - Démo - Rapport d’avancement \\
%
\rowcolor{green!50!black}
8 &
25/06 - 01/07 &
Début de la troisième itération \\
%
\rowcolor{green!50!black}
9 &
02/07 - 08/07 & \\
%
\rowcolor{green!50!black}
10 &
09/07 - 15/07 &
12/07 : Fin de la troisième itération - Démo \\
%
11 &
16/07 - 22/07 &
19/07 : Test bêta \\
%
12 &
23/07 - 29/07 &
26/07 : Revue de code \\
%
13 &
30/07 - 05/08 &
Remise du rapport final - Présentation du projet \\
%
\hline
\end{tabular}
		\label{tab.echeancier}
	\end{table}

	\subsection{Conditions critiques}
	La première condition critique est la mise en place des machines virtuelles sur les serveurs de l'UdeS qui hébergeront les différents services de l'application et sa base de données. Sans cette condition, le déploiement de l'application ne pourra pas se faire et le projet n'aura pas lieu d'être.

	La seconde condition critique est l'accès au service d'authentification mis en place par l'université afin d'assurer une sécurité équivalente à celle déjà existante. Ensuite, l'accès aux données du site \emph{Horarius} permettra de constater l'efficacité du système de notifications. \textcolor{blue}{Ainsi, si le projet sera indépendant des applications desquelles il tirera les données utiles d'un point de vue fonctionnel, son utilité dépendra en partie des services qui tireront partie du système de notifications}\marginpar{\emph{pas clair\dots}}.

	Enfin, la mise en place d'un serveur d'authentification interne à l'application qui assurera la connexion sur une période de temps plus longue que celle fournie par le CAS sera essentielle dans le sens où l'application devra fournir un service rapide sur toute la durée d'une session d’étude pour ne pas perdre de son intérêt.

	\subsection{Extrants, indicateurs d'achèvement et vérification}
	Le premier extrant de la réussite du projet est une connexion prolongée de l’utilisateur. En effet, lorsque l’utilisateur sera en mesure de se connecter qu’une fois par session et d’utiliser l’application sans perdre son accès, ce service sera considéré comme réussi. Bien-sûr, il ne faut pas supprimer l’application mobile ou se déconnecter.

	Le deuxième extrant est l’accès à l’horaire et éventuellement, aux notes et aux messages des professeurs les plus récents. L’utilisateur devra pouvoir utiliser l’application mobile sans qu'il n'y ait de différences avec les données sur le site web GEL.

	Finalement, la réception de notifications rapides signalant tout changement dans les différents éléments fournis par l’application est un extrant critique pour le succès du projet. L’utilisateur n'aura pas à aller sans cesse sur son téléphone pour être à jour dans les éléments transmis par son département.

	\subsection{Risques, déclencheurs et plan de contingence}
	\begin{table}[hp]
		\centering
		\renewcommand{\arraystretch}{1.5}
\begin{tabular}{P{0.25\textwidth}P{0.25\textwidth}P{0.25\textwidth}ll}
\hline
\bf Risque & \bf Déclencheurs & \bf Plan de contingence & \bf Sévérité & \bf Probabilité \\
\hline
\hline
Manque de main-d’oeuvre
& Un des développeurs a un accident grave, ou vient à manquer pour diverses raisons
& Les membres de l’équipe restants se réunissent pour répartir à nouveau les tâches qui étaient attribuées au développeur manquant
& Grave
& Moyen \\
%
Accès impossibles aux services de l’université
& Un problème technique majeur survient avec les serveurs de l’UdeS
& L’équipe prend les mesures adéquates pour informer les utilisateurs qu’un problème externe à l’application est survenu
& Grave
& Faible \\
%
Problème avec la gestion de version
& La license Gitlab utilisée pour la gestion de version est retirée ou n’est plus valide
& L’équipe explore les options de récupération possibles avec Gitlab
& Grave
& Faible \\
%
Perte de données dans Gitlab
& Défaillance dans les VMs fournies
& Toutes les \emph{issues} sont synchronisées avec un logiciel tiers (\emph{backup})
& Moyen
& Moyen \\
%
Echec aux tests d’intégration
& Un des modules de l’application ne peut pas être intégré au reste du projet
& L’équipe identifie les raisons de ce problème d’intégration et les développeurs en charge du module recommencent, ou d’autres développeurs se chargent de ce module
& Moyen
& Moyen \\
%
Maîtrise d’une technologie utilisée
& Les développeurs ne parviennent pas à maîtriser le \emph{framework} React Native
& L’équipe prend la décision de changer de \emph{framework} ou répartit à nouveau les tâches à d’autres développeurs
& Moyen
& Faible \\
\hline
\end{tabular}
		\caption{Concepts de risque par rapport au projet}
		\label{tab.risques}
	\end{table}