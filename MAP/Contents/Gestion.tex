\section{Gestion (stratégies retenues)}
	\subsection{Méthodes de mesure de l'avancement}
	La gestion de l’avancement du projet est faite à l’aide de la méthodologie SCRUM. Pour ce faire, on utilise le logiciel Gitlab. Ce logiciel nous permet de faire la gestion des tâches, la gestion des heures et de produire facilement les courbes nécessaires pour faire une gestion efficace de notre projet.
 
	Pour être en mesure de faire une gestion efficace lors de la session, l’équipe doit préalablement diviser le projet en plusieurs tâches et faire une estimation de temps nécessaire pour chaque tâche. Ainsi, l’équipe sait combien de d’heures estimées il reste avant chaque revu en fonction du temps déjà effectué.
 
	Définir adéquatement les tâches avant de commencer le projet est un point important pour être en mesure de faire une gestion efficace, mais l’assiduité des membres est nécessaire à la réussite de la gestion. En effet, il est important que les membres rentrent leurs heures travaillés dans l’application et qu’il s’assure que le projet avance comme prévu. Si l’équipe juge que le projet est en retard et que l’estimation n’était pas bonne, le remaniement des tâches est possible en cours de projet. 

	\subsection{Processus de gestion des décisions}
	Aucun membre n’a plus de pouvoir qu’un autre dans l’équipe, ainsi les décisions importantes se font lors des réunions d’équipe. Chaque membre travaillant sur cette section doit être minimalement présent. Idéalement, toute l’équipe doit être présente pour prendre la décision.
 
	Lors d’une prise de décision, il doit y avoir au minimum 4 membres de l’équipe. Le vote est à découvert et la résolution doit obtenir 50\% plus un pour être adapter.

	\subsection{Organisation de l'information}
	Chaque tâche doit avoir une description claire et précise pour permettre au membre de savoir en quoi elle consiste. Si jamais la tâche n’est pas claire, il en revient au membre à qui la tâche est assignée de la définir de façon adéquate. En effet, chaque tâche doit avoir une description assez précise pour permettre à n’importe quel membre de la reprendre.

	Pour la gestion des documents, comme les rapports et les graphiques on utilise un Google drive commun. On utilise Google doc pour l’écriture des rapports, ce qui permet à tous les membres de travailler sur le document en même temps.
 
	Pour la gestion du code, on utilise le système de gestion de versions GIT. On garde ainsi une histoire si jamais un pépin arrive. On utilise un Gitlab installé directement sur les machines de l’Université. 

	\subsection{Gestion et organisation de l'équipe}
	Étant donné que c’est un projet dans un cadre Universitaire, aucun membre de l’équipe est supérieur à un autre. Chaque membre à la responsabilité de fournir assez de temps et d’effort pour mener le projet à terme dans les temps établi par le cadre professoral.
 
	Toutefois, il y a certaines personnes qui sont responsable de suivre l’avancement du projet et de s’assurer que les membres de l’équipe n’oublient pas des parties importantes. Premièrement, un membre doit suivre la gestion du projet. Ce dernier, à la responsabilité de s’assurer que toutes les membres utilisent Gitlab convenablement. Il doit aussi s’assurer que les différentes courbes relier à la méthodologie SCRUM soient fait. Deuxièmement, chaque grande section du projet (BD, serveur, HTML, etc) doit avoir un responsable qui s’assure que la section est faite convenablement (code et tests) et dans les temps.

	\subsection{Organisation des rencontres}
	L’équipe prévoit une rencontre hebdomadaire lors des cours de projet. Ainsi, les membres de l’équipe ne sont pas obligés de se déplacer pour la rencontre d’équipe. Toutefois, les membres de l’équipe peuvent organiser une rencontre à d’autres moments dans la semaine s’il juge la réunion pertinente. Pour s’assurer que le plus de personne soit présente, il doit consulter les autres membres afin de déterminer l’heure et l’endroit.

	\subsection{Gestion et révision des objectifs}
	Lors de chaque réunion hebdomadaire, l’équipe doit se commettre à faire un certain nombre de tâche. Lors de la semaine suivante, chaque membre dit comment a été sa semaine, s’il a fini ses tâches et s’il a des bloquants pour la semaine à venir. Comme ça, on est capable de bien partager les tâches entre les membres pour s’assurer d’atteindre nos objectifs avant une remise.

	Aussi, on doit réserver un temps après chaque grosse remise pour faire une rétroaction d’équipe sur comment ça été. Ainsi, on est capable de sortir les points forts et les points faibles et de changer nos méthodes de faire pour obtenir de meilleurs résultats lors des autres remises.
