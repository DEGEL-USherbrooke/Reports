% Minimal Packages
\usepackage[french]{babel}
\usepackage[T1]{fontenc}
\usepackage[utf8]{inputenc}
\usepackage{microtype, lmodern}
\usepackage[hidelinks]{hyperref}
\usepackage{pdfpages}
\usepackage{palatino}

% Page Layout
\usepackage[top=2.54cm, bottom=2.54cm, right=3.17cm, left=3.17cm]{geometry}
\usepackage{
	setspace,
	titlesec,
	etoolbox,
	multirow,
	caption,
	multicol,
	subfig,
	float,
	makecell,
	pdflscape,
	rotating,
	enumerate,
	amssymb}

% Tiks Figures
\usepackage{xcolor, tikz}
\usetikzlibrary{babel} % Conciliation babel et tikz

% Table Fix
\usepackage{array}
\newcolumntype{P}[1]{>{\raggedright\arraybackslash}p{#1}}

% Mathematics
\usepackage{siunitx}
\sisetup{locale=FR}
\usepackage{
	amsmath,
	xfrac,
	commath,
	units,
	icomma,
	cancel}
\numberwithin{equation}{section}

% Lorem Ipsum
\usepackage{chngcntr} % wtf... aucune idée
\counterwithin{figure}{section}
\usepackage{lipsum} % Faux-texte
\newcommand{\lipsumC}{\lipsum[\value{lipsumCTR}] \stepcounter{lipsumCTR}}
\newcounter{lipsumCTR} \setcounter{lipsumCTR}{1}

% University Layout
\titlespacing*{\section}{0pt}{6pt}{6pt} % IEEE standard
\titlespacing*{\subsection}{0pt}{6pt}{6pt} % IEEE standard
\titlespacing*{\subsubsection}{0pt}{6pt}{6pt} % IEEE standard

\newcommand{\ohs}{\onehalfspacing} % IEEE standard
\renewcommand{\theequation}{\thesection-\arabic{equation}} % IEEE standard
\renewcommand{\thefigure}{\thesection-\arabic{figure}} % IEEE standard
\renewcommand{\thetable}{\thesection-\arabic{table}} % IEEE standard
\addto\captionsfrench{\def\tablename{\sc{Tableau}}} % Renommer les tableaux
\patchcmd{\thebibliography}{\section*{\refname}}{}{}{} % Renommer la bibliographie
\makeatletter
\renewcommand\paragraph{\@startsection{paragraph}{4}{\z@}%
                                    {1ex \@plus1ex \@minus.2ex}%
                                    {-1em}%
                                    {\normalfont\normalsize\bfseries}}
\setlength{\parindent}{0pt}

\makeatother
%: My Own Commands
\newcommand{\cip}[1]{#1}
\newcommand{\nomF}[1]{\textsc{#1}}
\newcommand{\code}[1]{\texttt{#1}}
\newcommand{\membre}[3]{\texttt{#3} & #1 \textsc{#2} \\}

\def \titreRapport {Rapport d'avancement}
\def \dateRemise {7 juin 2018}
\def \titreProjet {Projet DEGEL}
\def \numeroIteration {\# 1.0}

