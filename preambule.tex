% Minimal Packages
\usepackage[french]{babel}
\usepackage[T1]{fontenc}
\usepackage[utf8]{inputenc}
\usepackage{microtype, lmodern}
\usepackage[hidelinks]{hyperref}
\usepackage{palatino}

% Page Layout
\usepackage[top=2.54cm, bottom=2.54cm, right=3.17cm, left=3.17cm]{geometry}
\usepackage{
	setspace,
	titlesec,
	etoolbox,
	multirow,
	caption,
	multicol,
	subfig,
	float,
	makecell,
	pdflscape,
	rotating,
	enumerate,
	amssymb,
	comment}

% Tiks Figures
\usepackage{xcolor, tikz}
\usetikzlibrary{babel} % Conciliation babel et tikz

% Braces into Tables
\usetikzlibrary{decorations.pathreplacing,calc}

\newcommand{\tikzmark}[2][-3pt]{\tikz[remember picture, overlay, baseline=-0.5ex]\node[#1](#2){};}

\tikzset{brace/.style={decorate, decoration={brace}},
 brace mirrored/.style={decorate, decoration={brace,mirror}},
}

\newcounter{brace}
\setcounter{brace}{0}
\newcommand{\drawbrace}[3][brace]{%
 \refstepcounter{brace}
 \tikz[remember picture, overlay]\draw[#1] (#2.center)--(#3.center)node[pos=0.5, name=brace-\thebrace]{};
}

% Table Fix
\usepackage{array}
\newcolumntype{P}[1]{>{\raggedright\arraybackslash}p{#1}}

% Lorem Ipsum
\usepackage{chngcntr} % wtf... aucune idée
\counterwithin{figure}{section}
\usepackage{lipsum} % Faux-texte
\newcommand{\lipsumC}{\lipsum[\value{lipsumCTR}] \stepcounter{lipsumCTR}}
\newcounter{lipsumCTR} \setcounter{lipsumCTR}{1}

% University Layout
\titlespacing*{\section}{0pt}{18pt}{12pt} % IEEE standard
\titlespacing*{\subsection}{0pt}{12pt}{12pt} % IEEE standard
\titlespacing*{\subsubsection}{0pt}{12pt}{12pt} % IEEE standard

\newcommand{\ohs}{\onehalfspacing} % IEEE standard
\renewcommand{\theequation}{\thesection-\arabic{equation}} % IEEE standard
\renewcommand{\thefigure}{\thesection-\arabic{figure}} % IEEE standard
\renewcommand{\thetable}{\thesection-\arabic{table}} % IEEE standard
\addto\captionsfrench{\def\tablename{\sc{Tableau}}} % Renommer les tableaux
\patchcmd{\thebibliography}{\section*{\refname}}{}{}{} % Renommer la bibliographie
\setlength{\parindent}{0pt}

%: My Own Commands
\newcommand{\cip}[1]{#1}
\newcommand{\nomF}[1]{\textsc{#1}}
\newcommand{\code}[1]{\texttt{#1}}
\newcommand{\membre}[3]{\texttt{#3} & #1 \textsc{#2} \\}

\def \titreRapport {Rapport d'éthique}
\def \dateRemise {26 juin 2018}
\def \titreProjet {Projet DEGEL}
\def \numeroIteration {\# iteration}

