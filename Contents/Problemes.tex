\section{Problèmes rencontrés}
	\subsection{Backend}
	\paragraph{Horarius} L'accès au JSON d'horarius s'avère plus complexe que prévu. La version iCal n'est pas présentée de la même manière, même si convertie en JSON, ce qui nécessiterait davantage de manipulation de données.
	
	\subsection{Mobile}
	L'offre de frameworks de calendriers en React Native avoisine la nullité. Cependant, l'équipe ne veut pas faire la conception calendrier à partir de rien, car c'est un défi énorme. Il y a beaucoup de \emph{edge cases} à penser et il est très difficile de faire du beau UI sans un designer très expérimenté. Nous tenterons jusqu'au bout d'utiliser les \emph{plugins} qui existent et de les étendres avant de faire notre propre implémentation.
	
	\subsection{Gestion}
	L'équipe aura avantage à instaurer des normes à suivre sur Git. Se rapprochant de la fin du baccalauréat, chaque membre de l'équipe possède sa propre expérience en entreprise de l'outil et sa propre définition de \og bonnes pratiques \fg, ce qui mène parfois à de petits débats.