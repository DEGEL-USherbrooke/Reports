\section{Définition des axes innovants et parties prenantes}
	\subsection{Choix du cas}
	Nous avons sélectionné le premier cas, car il répond à la plus grande demande sociale des étudiants. En effet, sans notifications, il se crée de la friction lors de changements d'horaire de dernière minute.
	
	\subsection{Axes innovants du cas choisi}
	\begin{enumerate}
		\item Application mobile vs site web
		\item Notifications vs aucune notification
		\item Connexion longue durée (semestrielle) vs Connexion courte durée (chaque connexion)
	\end{enumerate}
	
	\subsection{Parties prenantes du cas choisi}
		\subsubsection{Parties prenantes directes}
		\begin{itemize}
			\item Étudiants de GEGI
			\item Équipe professorale
			\item Responsables d'Horarius
			\item Équipe de notifications
			\item Responsable de la sécurité du réseau
		\end{itemize}
		
		\subsubsection{Parties prenantes indirectes}
		\begin{itemize}
			\item Techniciens informatiques
			\item Coordonateur académique
			\item Responsable du CAS
			\item Magasins d'applications mobiles
			\item Entreprise de serveurs de notifications (Expo)
		\end{itemize}
	