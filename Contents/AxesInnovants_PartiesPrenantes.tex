\section{Définition des axes innovants et parties prenantes}
	\subsection{Choix du cas}
	Nous avons sélectionné le premier cas (accès sur mobile aux services du département GEGI), car il répond à la plus grande demande sociale des étudiants. En effet, sans notifications, il se crée de la friction lors de changements d'horaire de dernière minute.
	
	\subsection{Axes innovants du cas choisi}
	\begin{enumerate}
		\item Application mobile vs site web
		\item Notifications vs aucune notification
		\item Authentification longue durée (semestrielle) vs authentification courte durée (à chaque connexion)
	\end{enumerate}
	
	\subsection{Parties prenantes du cas choisi}
		\subsubsection{Parties prenantes directes}
		\begin{itemize}
			\item Étudiants du département de GEGI
			\item Équipe professorale du département de GEGI
			\item Responsables du système d'horaire \emph{Horarius}
			\item Équipe de notifications (autre équipe de projet S6)
			\item Responsable de la sécurité du réseau \emph{aerius}
		\end{itemize}
		
		\subsubsection{Parties prenantes indirectes}
		\begin{itemize}
			\item Techniciens informatiques du département de GEGI
			\item Coordonateur académique du département de GEGI
			\item Responsable du CAS (\emph{Central Authentification System})
			\item Magasins d'applications mobiles (App Store, Google Play)
			\item Entreprise de serveurs de notifications (Expo)
		\end{itemize}
	