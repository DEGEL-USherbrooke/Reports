\subsubsection{Backend}
% Auth Oauth2, être capable de faire un DIFF sur un JSON d'horaire, faire un API pour recevoir les API

%\paragraph{Authentification} Posséder une méthode d'authentification avec OAuth2 fonctionnelle.
%
%\paragraph{DIFF} Être en mesure d'effectuer nous-mêmes un DIFF sur un fichier JSON enregistré localement.
%
%\paragraph{API} Concevoir un API pour recevoir les notifications de l'autre équipe.

\subsubsection{Mobile}
% Corriger éléments UI, faire itération 2 du calendrier custom, intégration avec backend, notifications

%\paragraph{UI} Corriger certains éléments incohérents de l'UI (titres dédoublés, barres de navigation).
%
%\paragraph{Calendrier} Produire la 2\ieme{} itération du calendrier qui se connectera au serveur.
%
%\paragraph{Authentification} Commencer à intégrer l'authentification du \emph{backend} dans l'application mobile.
%
%\paragraph{Notification} Développer des notifications de base sur l'application.
%
%\paragraph{I18n} Développer l'internationalisation sur l'application pour se faire approuver sur l'App Store.

\subsubsection{Gestion}
% Peaufiner le pointage des tâches (temps), établir standards GIT

%\paragraph{Pointage} Peaufiner le pointage des tâches en ce qui a trait au temps requis pour chacune.
%
%\paragraph{Git} Établir des standards pour l'utilisation de Git.