\section{Gestion}
    \subsection{Sommaire de l'avancement}
    Au début du projet, l'équipe a décidé de faire son suivi de gestion sur Gitlab. C'était un nouvel outil de gestion pour tous les membres de l'équipe. Avec la version d'essai de 30 jours, le logiciel répondait aux besoins de l'équipe. Malheureusement, lorsque la période d'essai fut terminée, la version gratuite de Gitlab n'offrait aucun outil de gestion intéressant. Ainsi, l'équipe décida d'utiliser le logiciel de gestion Jira. Ce dernier était très bien fait pour suivre la gestion et l'avancement du projet. Le changement d'un outil de gestion vers un autre s'est très bien fait, et n'a aucunement affecté le bon déroulement projet.

    Pour faire le suivi de l'avancement, l'équipe se rencontrait chaque jeudi pour connaître l'avancement des tâches du sprint courant. En effet, chaque membre parlait de l'avancement de ses tâches en cours et des problèmes qu'il a rencontrés. Ainsi, l'équipe était en mesure de transférer des ressources ou des tâches si un membre avait trop de tâches. Lorsqu'un jeudi tombait en fin de sprint, l'équipe se décidait des tâches présentes dans le prochain sprint et discutait de ce qui avait bien ou mal été lors du dernier sprint.
    
    Chaque grande section du projet avec un responsable pour suivre l'avancement des tâches. De cette manière, chaque section avait un responsable qui s'assurait qu'à long terme, tout serait prêt pour les différentes remises. Son rôle n'était pas de faire toute la section, mais plutôt de s'assurer que toutes les ressources nécessaires étaient à la disposition des membres.

    \subsection{Sommaire des heures travaillées par rapport aux heures planifiées}
    Étant donné que les membres n'avaient pas le même niveau de compétence, l'équipe a décidé de faire le suivant de l'avancement en points et non en heures. De cette manière, le nombre d'heures travaillées n'était pas une mesure prioritaire pour l'équipe.

    Chaque membre avait environ le même nombre de points à chaque sprint, ce qui veut dire que l'avancement du projet dépendait davantage d'une complexité que d'un nombre d'heures travaillées. Ce choix a été fait pour pallier les différences de niveaux techniques au début de la session. De plus, étant donné qu'un membre avait déjà fait ce genre de projet, évaluer en termes de complexité était plus facile qu'en termes de durée. Nos évaluations s'avéraient ainsi plus précises.

    Néanmoins, les points étaient reliés à une intervalle d'heures selon un barème (tableau~\ref{tab.bareme}) établi par l'équipe afin d'avoir une idée générale du temps de travail effectué. Généralement un sprint contenait 50 à 60 points, représentant 100 à 150 heures de travail.
    
    \begin{table}[bhp]
        \centering
        \caption{Barème de gestion des points}
        \begingroup
        \renewcommand{\arraystretch}{1}
        \begin{tabular}{ll}
            \hline
            \bf Nombre de points & \bf Durée (heures) \\
            \hline
            \hline
            1 & 0,5 - 1 \\
            2 & 3 - 4 \\
            3 & 5 - 8 \\
            4 & 9 - 12 \\
            5 & 13+ \\
            \hline
        \end{tabular}
        \endgroup
        \label{tab.bareme}
    \end{table}

    \subsection{Sommaire des revues de code et de design}
    Les revues de code et de design ont été faites au fur et à mesure qu'elles devaient avoir lieu. En effet, notre outil de gestion Gitlab nous a permis de suivre l'avancement du projet en temps réel. Lorsqu'un membre de l'équipe voulait joindre sa branche dans la branche principale, son code devait être passé en revue et être approuvé par au moins un autre membre de l'équipe. Lorsqu'un membre faisait une revue de code ou de design, ce dernier pouvait facilement poser des questions à l'aide de notre outil de gestion de code. De cette manière, l'équipe arrivait à avoir la meilleure pratique méthodique en fonction de l'expérience de tous les membres.
 
    Les revues de design ont surtout eu lieu au début de la session. Une fois le standard de code établi, nous utilisions toujours ce dernier pour ajouter une nouvelle fonctionnalité au projet. Aussi, les revues de code étaient plus facilement acceptées à la fin de projet, car tous utilisaient très bien les standards de l'équipe.

    \subsection{Suivi des bugfixes}
    La gestion des bugs a été très efficace tout au long du projet. Lorsqu'un bug était trouvé, un membre de l'équipe était assigné à la tâche de régler le bug. La personne assignée était très souvent la personne qui avait écrit le code contenant le bug. Ainsi, elle était déjà au courant du code et de la logique implémentée. La personne responsable du bug avait comme devoir de s'assurer que le bug était réglé le plus rapidement possible et de tester que le problème avait été résolu. Aussi, dès le début, l'équipe a établi un standard établissant que chaque bug devait avoir un test unitaire pour s'assurer que le bug ne revienne pas dans le futur.

    \subsection{Sommaire des objectifs du projet}
    Pour le projet de S6, l'équipe avait pour mission de créer une application mobile afin de permettre aux étudiants d'accéder aux services du site web GEL via un téléphone mobile. Ce projet visait à apporter de la mobilité et de l'accessibilité à l'utilisateur. L'application est un premier pas permettant au département de génie électrique et de génie informatique de s'adapter aux nouvelles technologies tout en réduisant le taux d'absence ou de retard en cas de changement d'horaire.

    Plusieurs objectifs étaient visés en ce qui concerne ce projet. Tout d'abord, l'utilisateur devait avoir une application mobile lui permettant de se connecter qu'une seule fois par session et d'accéder à la dernière version de son horaire. Celui-ci devait aussi avoir accès à un service de notifications avec plusieurs options afin d'être au courant de tout changement dans l'horaire.

    Un second objectif était d'offrir le même service pour d'autres services. Si le temps nous le permettait, l'équipe devait offrir un service identique s'appliquant aux messages écrits par les professeurs sur la page d'accueil et pour les notes. Cet objectif a été partiellement abandonné au profit d'une intégration à l'équipe de notifications.

    \pagebreak
    \subsection{Extrants, indicateurs d'achèvement atteints et processus de vérification}
    Le premier extrant de la réussite du projet était la connexion prolongée de l'utilisateur. Le système actuel comprend une authentification sur le serveur CAS de l'Université de Sherbrooke, tel que requis. Puisque l'utilisateur est en mesure de se connecter qu'une seul fois par session et d'utiliser l'application sans perdre son accès, ce service est donc considéré réussi.

    Le deuxième extrant était l'accès à l'horaire et, éventuellement, aux notes et aux plus récents messages des professeurs. Actuellement, on peut utiliser l'application mobile dans le but de consulter son horaire. Le système modulaire mis en place permet une intégration éventuelle des autres fonctionnalités. Ce deuxième extrant est donc aussi considéré réussi.

    Finalement, la réception rapide de notifications signalant tout changement dans les différents éléments fournis par l'application était un extrant critique pour le succès du projet. Celui-ci est aussi un succès puisque l'utilisateur peut recevoir des notifications sur son téléphone dans un délai maximal de 5~minutes après un changement à l'horaire et lorsque des services externes publient des notifications par l'équipe des notifications.
    
    
    
    