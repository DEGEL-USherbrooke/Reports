\section{Introduction}
Toute université possède diverses plateformes destinées à connecter le corps professoral à la communauté étudiante. Ces plateformes rassemblent par exemple l'horaire, les résultats académiques, les travaux, etc. Au sein du département de génie électrique et de génie informatique de l'Université de Sherbrooke, la plateforme Horarius sert à afficher un horaire individuel aux étudiants, lequel est administré par les coordonnateurs du programme. Sur un fureteur d'ordinateur, l'horaire s'affiche sous un format lisible et clair; sur mobile, le même contenu peine à s'afficher sur un écran rikiki. Qui plus est, aucune notification n'est envoyée en cas de changement d'horaire, et il faut entrer ses identifiants à chaque connexion, ce qui constitue un irritant supplémentaire.

Dans le cadre du cours de conception de système distribué, l'équipe a décidé de s'attaquer à ces trois problématiques en créant une application mobile qui afficherait l'horaire individuel dans un format plus adapté à cette taille d'écran, qui enverrait une notification lorsqu'un changement à l'horaire survient, et qui éviterait à l'utilisateur d'avoir à se reconnecter pour la durée d'un trimestre entier. Au moment d'écrire ces lignes, l'application est fonctionnelle, disponible sur le Play Store et en attente d'approbation par les évaluateurs de l'App Store.

Ce rapport décrit la démarche entreprise pour réaliser le projet de conception dans le temps alloué. On y décrit les outils de gestion utilisés, on y explique les décisions prises quant aux  technologies de développement possibles, on y décrit les méthodes de test appliquées tout au long du projet, et enfin, on porte un regard post-mortem sur le travail réalisé. Des recommandations s'ensuivent pour guider toute envie d'amener l'application plus loin.