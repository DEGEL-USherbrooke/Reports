\section{Introduction}
Au début du mois d'avril 2018, Mark Zuckerberg, fondateur de Facebook, se présentait devant le congrès des États-Unis pour expliquer la situation dans laquelle se trouvait l'une des entreprises les plus fortunées et connues de la planète. Facebook est loin d'être un cas isolé. Bon nombre d'application contiennent des imperfections résultant d'une mauvaise conception, les rendant ainsi exploitables par quiconque sachant comment s'y prendre. Ces tournées dans les médias montrent bien que la sécurité du publique prime, peu importe le diplôme que l'on a en poche. \\

Dans le cadre du cours d'éthique, on nous demande d'évaluer notre projet de session en effectuant une démarche responsable -- par rapport à l'éthique, à l'acceptabilité sociale et à l'usage -- afin d'améliorer ledit projet. Notre équipe, DEGEL, conçoit durant la session une application dédiée à l'horaire du département de génie électrique et génie informatique. D'apparence simple, l'application comporte plus d'un enjeu, notamment en terme de développement, d'implémentation et d'utilisation. Il convient donc de les aborder correctement. \\

Le présent rapport contient notre démarche responsable visant l'amélioration du projet. On y explique plus en détail le concept en soi et les objectifs fixés. La définition des axes innovants et des parties prenantes dresse un profil général de ce qu'apporte l'application aux groupes visés directement et indirectement. Sous forme de tableaux, les impacts et enjeux par rapport à l'application sont détaillés de manière à montrer clairement le questionnement de l'équipe quant aux innovations de l'application. Enfin, la veille documentaire est établie selon la priorité des enjeux et apporte matière à réflexion sur le développement de l'application. Le rapport se conclut sur les moyens d'intégration définis par l'équipe suivant la mise au point de la veille documentaire.