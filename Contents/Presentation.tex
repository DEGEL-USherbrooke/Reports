\section{Présentation}
	% Contexte
	\subsection{Contexte général}
	Notre concept se nomme \emph{DEGEL}. Il s'agit d'une application mobile iOS et Android.

	% Definition
	\subsection{Définition du concept}
	
		% 1er cas d'usage
		\subsubsection{1\ier{} cas d'usage} Il vise les étudiants de GEGI et sert à accéder aux services du département de GEGI. Il vise à apporter simplicité, mobilité et accessibilité à l'utilisateur. En termes d'implémentation sociale, les contextes visés sont les locaux de la faculté de génie du campus de l'Université de Sherbrooke.
		
		% 2e cas d'usage
		\subsubsection{2\ieme{} cas d'usage} Il vise les étudiants et l'équipe professorale du département de GEGI, et sert à établir une plateforme d'échange social entre ceux-ci. Il vise à améliorer la communication en temps réel. En termes d'implémentation sociale, les contextes visés sont les espaces virtuels créés par la plateforme. \\
	
			Pays concerné : Canada \textcolor{white}{et le Québec libre !}
	
		% Finalite
		\subsubsection{Finalité du projet}
		Nous faisons ce projet pour adapter aux nouvelles technologies la consommation de l'information sur le site \emph{gel}, en vue de réduire les délais de communication, les taux d'absentéisme en cas de changement d'horaire, etc. Ultimement, la finalité de notre projet est d'amener sur mobile ce qui existe déjà sur le site web du département, ce qui répond à la demande sociale des étudiants. \\
		
		Nos moteurs (personnels) pour le réaliser : 
		\begin{itemize}
			\item Mettre en \oe uvre de bonnes pratiques de programmation
			\item Apprendre un nouveau langage de programmation
			\item Développer une application mobile
		\end{itemize} \vspace{\baselineskip}
		
		Nos raisons communes pour le réaliser :
		\begin{itemize}
			\item Éviter d'avoir à se connecter chaque fois au site
			\item Recevoir des notifications en cas de changement
		\end{itemize} \vspace{\baselineskip}
		
		Nous pourrions éventuellement intégrer le cas 2 au cas 1 afin d'offrir une expérience plus complète à l'utilisateur. Dans un esprit de collaboration, considérant qu'une autre équipe s'occupe du système de messagerie, nous pourrions potentiellement les intégrer dans notre application.
		