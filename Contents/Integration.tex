\section{Intégration}
\subsection{Défis prioritaires}
\begin{enumerate}
	\item \textbf{Addiction:} Tel que nous avons mentionné plusieurs reprises, le phénomène d'addiction au téléphone est très prépondérante et problématique chez la population qui représente nos utilisateurs cibles.
	\item \textbf{Sécurité:} Comme nous allons travailler avec des informations sensibles et potentiellement confidentielles des utilisateurs, il est primordial de se soucier de l'aspect de la sécurité qui doit se trouver au centre de notre design.
	\item \textbf{Inclusion:} C'est le défi qui a retenu notre attention dans la veille documentaire. Nous croyons que c'est le devoir de tous de faire un effort en cette direction. 
\end{enumerate}

\subsection{Enrichissement du projet}
Qu’est ce que ça pourrait modifier / venir enrichir dans votre projet si vous deviez concrètement les prendre en compte?
\begin{enumerate}
	\item \textbf{Addiction:}
	\begin{itemize}
		\item Réglage pour désactiver les notifications
		\item Permettre à l'utilisateur de déterminer les heures de notification (pas pendant la nuit par exemple)
		\item Réglage pour choisir les notifications que l'utilisateur souhaite recevoir
		\item Interface qui ne capture pas l'attention (peu d'animations, centrée sur le contenue)
	\end{itemize}
	\item \textbf{Sécurité:}
	\begin{itemize}
		\item Toujours commencer par authentifier les utilisateurs avec le serveur d'authentification de l'Université
		\item Toujours vérifier les permissions de l'utilisateur courant avant le de laisser faire une action
		\item Redemander à l'utilisateur de se connecter à intervalle régulier 
		\item Être capable de révoquer les accès d'un utilisateur
		\item Permettre à l'utilisateur de déconnecter tous ses appareils d'un clique
	\end{itemize}
	\item \textbf{Inclusion:}
	\begin{itemize}
		\item Interface neutre, non genrée avec tout ce que ça implique pour le choix des couleurs, des formes et du langage. 
		\item Ne pas demander ni afficher d'information liée au genre ou au sexe à l'utilisateur
		\item Intégration des outils d'accessibilité pour les personnes malvoyantes
		\item Consulter des membres de ces différentes populations afin de s'assurer que l'application correspond à leurs attentes et est utilisable par tous
	\end{itemize}
\end{enumerate}

Avez-vous appris de nouvelles choses / été surpris en faisant cette démarche pour votre projet?
\subsection{Apprentissages}
La démarche éthique nous a permis de prendre conscience de plusieurs enjeux et défis à surmonter pour la réalisation de notre projet. Trois aspects ont particulièrement retenu notre attention.

Premièrement, l'inclusion de tous les utilisateurs. Ce n'est pas un aspect facile à traiter puisqu'il n'y a pas une bonne façon de faire pour développer des applications inclusives et encore moins de définition claire de ce que cela veut dire. Par contre, nous avons vraiment compris l'importance de cet enjeu en 2018 et de la nécessité de le prendre en compte dans le design. Nous n'avions par exemple pas du tout pensé aux personnes malvoyantes avant d'effectuer ce travail. 

Deuxièmement, le problème d'addiction aux téléphones et aux notifications. Même si nous savions que cela existait, nous ne nous doutions pas de l'ampleur du problème actuel que certains vont même jusqu'à qualifier de problème de santé publique majeur. En tant que développeur, il est très facile d'y contribuer puisque nous souhaitons tous développer de belles applications interactives qui capturent l'attention de nos utilisateurs. Une démarche éthique permet d'éviter ce piège et de trouver un équilibre acceptable entre convivialité et capture d'attention.

Troisièmement, nous n'avions pas idée du nombre de parties prenantes qui sont acceptées par un projet de très modeste envergure comme le nôtre.  Sans cette démarche, il aurait été facile d'en oublier ou d'en négliger et cela aurait pu mener à l'échec du projet. Cela nous as particulièrement marqué puisque pour nos projets futurs de plus grande envergure, cette démarche devient d'autant plus essentielle.

