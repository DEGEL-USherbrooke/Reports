\section{Intégration}

\subsection{Défis prioritaires}
\begin{description}
	\item[Addiction] Tel que nous l'avons mentionné à plusieurs reprises, le phénomène d'addiction au téléphone est une situation très prépondérante et problématique chez la population qui représente nos utilisateurs cibles.
	\item[Sécurité] Comme nous allons travailler avec des informations sensibles et potentiellement confidentielles des utilisateurs, il est primordial de se soucier de l'aspect de la sécurité, celle-ci devant se trouver au centre de notre design.
	\item[Inclusion] C'est le défi qui a retenu le plus notre attention dans la veille documentaire. Nous croyons que c'est le devoir de tout un chacun de faire un effort en cette direction. 
\end{description}

\subsection{Enrichissement du projet}
\begin{description}
	\item[Addiction] \hfill
		\begin{itemize}
			\item Instauration d'un réglage pour désactiver les notifications
			\item Permettre à l'utilisateur de déterminer les heures de notifications (\textit{e.g.} pas pendant la nuit)
			\item Instauration d'un réglage pour choisir les notifications que l'utilisateur souhaite recevoir
			\item Design d'interfaces ne capturant pas l'attention (peu d'animations, centrées sur le contenu)
		\end{itemize}
		
	\item[Sécurité] \hfill
		\begin{itemize}
			\item Toujours débuter par authentifier les utilisateurs avec le serveur dédié de l'Université
			\item Toujours vérifier les permissions de l'utilisateur courant avant de le laisser faire une action
			\item Redemander à l'utilisateur de se connecter à intervalle régulier 
			\item Être capable de révoquer les accès d'un utilisateur
			\item Permettre à l'utilisateur de déconnecter tous ses appareils d'un clic
		\end{itemize}
		
	\item[Inclusion] \hfill
		\begin{itemize}
			\item Interface neutre, non genrée (choix des couleurs, des formes et du langage)
			\item Ne pas demander ni afficher d'information liée au genre ou au sexe à l'utilisateur
			\item Intégration des outils d'accessibilité pour les personnes malvoyantes
			\item Consulter des membres de ces différents groupes afin de s'assurer que l'application correspond bien à leurs attentes et est utilisable par tous
		\end{itemize}
\end{description}

\pagebreak
\subsection{Apprentissages}
La démarche éthique nous a permis de prendre conscience de plusieurs enjeux et défis à surmonter pour la réalisation de notre projet. Trois aspects ont particulièrement retenu notre attention. \\

Premièrement, l'inclusion de tous les utilisateurs. Ce n'est pas un aspect facile à traiter puisqu'il n'y a pas une bonne façon de faire pour développer des applications inclusives et encore moins de définition claire de ce que cela veut dire. Par contre, nous avons vraiment compris l'importance de cet enjeu en 2018 et de la nécessité de le prendre en compte dans le design. Nous n'avions par exemple pas du tout pensé aux personnes malvoyantes avant d'effectuer ce travail. \\

Deuxièmement, le problème d'addiction aux téléphones et aux notifications. Même si nous savions que cela existait, nous ne nous doutions pas de l'ampleur du problème actuel que certains vont même jusqu'à qualifier de problème de santé publique majeur. En tant que développeur, il est très facile d'y contribuer puisque nous souhaitons tous développer de belles applications interactives qui capturent l'attention de nos utilisateurs. Une démarche éthique permet d'éviter ce piège et de trouver un équilibre acceptable entre convivialité et capture d'attention. \\

Troisièmement, nous n'avions pas idée du nombre de parties prenantes qui sont acceptées par un projet de très modeste envergure comme le nôtre.  Sans cette démarche, il aurait été facile d'en oublier ou d'en négliger et cela aurait pu mener à l'échec du projet. Cela nous a particulièrement marqués puisque pour nos projets futurs de plus grande envergure, cette démarche devient d'autant plus essentielle.

