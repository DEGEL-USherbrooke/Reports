\section{État du sprint}
	
	\subsection{Backend}
	% commencer la comm acec CAS (abandon pour auth plus sécuritaire OAuth2), monter le projet de base (bonnes pratiques), CI, CD Docker. Setup des frameworks  (spring, ORM, Kotlin, feign)

	\paragraph{Projet de base} La structure du projet \emph{Backend} a été choisie et le projet de base a été monté. Les frameworks utilisés tels que Hibernate, FlywayDB, Feign, Swagger et Spring ont été configurés.

	\paragraph{Sécurité} La première tentative d'authentification avec le CAS a été développée en utilisant la librarie Spring. Le service a été passé en HTTPS avec \emph{Letsencrypt} afin de protéger les informations des utilisateurs.
	
	\paragraph{Gitlab} Une installation pleinement fonctionnelle de GitLab a été mise en place. L'équipe utilise déjà l'intégration continue (CI) et le déploiement continu (CD) afin d'automatiser au maximum le roulement des tests et la mise à jour du serveur.
	
	\paragraph{Docker} Le projet roule entièrement dans Docker et peut être facilement déployé avec le \emph{docker-compose.yml} sur toute machine possédant Docker.
		
	\subsection{Mobile}
	% application de base (bonnes pratiques, hiérarchie des fichiers), trouver un framework pour l'horaire, interagir avec une webview pour l'auth CAS, la page de settings

	\paragraph{Application de base} En vue de la démonstration à faire, l'équipe a développé une application offrant des services de base (vue horaire, vue mensuelle, page de réglages).
	
	\paragraph{Framework d'horaire} L'équipe a discuté sur le choix du framework d'horaire, notamment sur le choix entre un framework \emph{custom} ou non. S'il est \emph{custom}, comment le faire ? Sinon, lequel choisir ?
	
	\paragraph{Webview} L'authentification avec le serveur CAS se déroulera via une webview directement dans l'application. L'équipe s'est donc penchée sur l'intégration et les interactions possibles avec ce qu'il y a derrière une webview quant à l'authentification (récupération de tokens, de cookies, etc.).	
	
	\paragraph{Réglages} Une première page de réglages est développée sur l'application et prête pour la démonstration. Il s'agit encore d'une version préliminaire.
	
	\subsection{Gestion}
	% setup GITLab pour la gestion (abandon pour manque de fonctions). JIRA désormais, reports collaboratif sur Gitlab avec CI

	\paragraph{GitLab} La plateforme GitLab est abandonnée en ce qui concerne la gestion des tâches en raison des limitations de sa version gratuite (une seule personne assignée, un seul réviseur, graphes manquants).
	
	\paragraph{JIRA} La plateforme JIRA remplace GitLab pour la gestion des tâches. Elle offre exactement ce que l'équipe recherchait pour (entre autres) regrouper les tâches en catégories et en sprints.
	
	\paragraph{Rapports} GitLab demeure la plateforme de rédaction des rapports en \LaTeX. Désormais, chaque personne peut écrire sa partie de rapport et la pusher dans GitLab sans avoir à compiler localement. GitLab s'occupe de générer le PDF.
	
	
