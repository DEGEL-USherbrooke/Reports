\subsubsection{Backend}

\paragraph{Authentification} TODO Emile

\paragraph{HTTPS} Le service (lequel ?) est passé au HTTPS.

\paragraph{Réglages} La base de données des réglages est créée, ainsi que son API.

\paragraph{Droits d'accès à l'horaire} Échec fatal \rWalley

\paragraph{Base de données de l'horaire} TODO What do we have...
	
\subsubsection{Mobile}

\paragraph{Internationalisation} Offrir l'application en plusieurs langues s'avère beaucoup plus complexe que prévu étant donné le manque de synchronisme entre les composantes.

\paragraph{Incohérences UI} Des incohérences au niveau de l'UI ont été corrigées avec succès. 

\paragraph{Icône} L'application possède maintenant une icône unique et distinctive.

\paragraph{Notifications} Un registre des notifications existe désormais pour\dots %TODO Register

\paragraph{CAS} L'interface de CAS est désormais intégrée dans une webview.

\paragraph{Calendrier} La version 2 du calendrier est fonctionnelle.

\subsubsection{Gestion}

\paragraph{GitLab} La licence GitLab a pris fin, donc l'équipe est sous la version gratuite et limitée de la plateforme. La licence pour établissements d'enseignement tarde étant donné le trop grand nombre d'institutions intéressées.

\paragraph{JIRA} L'équipe rentre ses heures correctement dans JIRA, ce qui améliore l'allure des courbes.

\paragraph{Rapports} Toute l'équipe contribue aux rapports via GitLab sans avoir nécessairement besoin d'une distribution \LaTeX{} locale.


