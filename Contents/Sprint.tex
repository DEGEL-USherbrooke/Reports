\subsubsection{Backend}

	\paragraph{Authentification} Tout est terminé, hormis la déconnexion et la révocation de tokens.
	
	\paragraph{HTTPS} Le serveur est en HTTPS.
	
	\paragraph{Réglages} La base de données des réglages est créée, ainsi que son API.
	
	\paragraph{Droits d'accès à l'horaire} Il faudra passer par un iCal et demander directement à l'utilisateur de rentrer son lien iCal. \textcolor{white}{\rWalley}
	
	\paragraph{Base de données de l'horaire} Terminé en ce qui concerne la récupération de l'horaire, le storage et la différence entre chaque version de l'horaire.
	
\subsubsection{Mobile}

	\paragraph{Internationalisation} Après quelques petits bugs, cette fonctionnalité est en place. Il y a que les langages doivent être chargés de manière synchrone, avant le chargement des composants.
	
	\paragraph{Incohérences UI} Des incohérences au niveau de l'UI ont été corrigées avec succès. 
	
	\paragraph{Icône} L'application possède maintenant une icône unique et distinctive.
	
	\paragraph{Notifications} Il manque le backend par rapport au \emph{register} des notifications.
	
	\paragraph{CAS} L'interface de CAS est désormais intégrée dans une webview.
	
	\paragraph{Calendrier} La version 2 du calendrier est fonctionnelle.

\subsubsection{Gestion}

	\paragraph{GitLab} La licence GitLab a pris fin, donc l'équipe est sous la version gratuite et limitée de la plateforme. La licence pour établissements d'enseignement tarde étant donné le trop grand nombre d'institutions intéressées.
	
	\paragraph{JIRA} L'équipe manque de rigueur quant à la mise à jour des tâches, ce qui réduit en miettes le \emph{burndown chart}.
	
	\paragraph{Rapports} Toute l'équipe contribue aux rapports via GitLab sans avoir nécessairement besoin d'une distribution \LaTeX{} locale.
	
	
