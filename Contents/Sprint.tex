\subsubsection{Backend}
% commencer la comm acec CAS (abandon pour auth plus sécuritaire OAuth2), monter le projet de base (bonnes pratiques), CI, CD Docker. Setup des frameworks  (spring, ORM, Kotlin, feign)

\paragraph{Projet de base} La structure du projet \emph{backend} a été choisie et le projet de base a été monté. Les frameworks utilisés tels que Hibernate, FlywayDB, Feign, Swagger et Spring ont été configurés.

\paragraph{Sécurité} La première tentative d'authentification avec le CAS a été développée en utilisant la librairie Spring. Le service a été passé en HTTPS avec \emph{Let's Encrypt} afin de protéger les informations.

\paragraph{Gitlab} L'équipe utilise déjà l'intégration continue (CI) et le déploiement continu (CD) afin d'automatiser au maximum le roulement des tests et la mise à jour du serveur.

\paragraph{Docker} Le projet roule entièrement dans Docker et peut être facilement déployé avec le \emph{docker-compose.yml} sur toute machine possédant Docker.
	
\subsubsection{Mobile}
% application de base (bonnes pratiques, hiérarchie des fichiers), trouver un framework pour l'horaire, interagir avec une webview pour l'auth CAS, la page de settings

\paragraph{Application de base} Une vue \emph{calendrier} est fonctionnelle avec des données de test enregistrées dans le téléphone. Une vue \emph{paramètres} possède des boutons de configuration.

\paragraph{Framework d'horaire} L'équipe a eu plusieurs rencontres de conception pour décider si un \emph{plugin} (et lequel) allait être utilisé ou non pour faire le calendrier. 

\paragraph{Webview} L'authentification avec le serveur CAS se déroulera via une \emph{webview} directement dans l'application. L'application récupérera les données (tokens) directement dans la \emph{webview}.

%\paragraph{Réglages} Une première page de réglages est développée sur l'application et prête pour la démo.

\subsubsection{Gestion}
% setup GITLab pour la gestion (abandon pour manque de fonctions). JIRA désormais, reports collaboratif sur Gitlab avec CI

\paragraph{GitLab} La plateforme GitLab est abandonnée en ce qui concerne la gestion des tâches en raison des limitations de sa version gratuite.

\paragraph{JIRA} La plateforme JIRA remplace GitLab pour la gestion des tâches. Elle offre exactement ce que l'équipe recherchait pour (entre autres) regrouper les tâches en catégories et en sprints.

\paragraph{Rapports} GitLab demeure la plateforme de rédaction des rapports en \LaTeX. Désormais, chaque personne peut écrire sa partie de rapport et la \emph{pusher} dans GitLab sans avoir à compiler localement.


