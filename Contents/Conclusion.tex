\section{Recommandations et conclusions}
Suivant la finalité d'un projet, il est nécessaire de réfléchir à certaines recommandations afin de prévoir la continuité du projet. La première recommandation, qui est probablement la plus importante, est la création d'«endpoints» avec les services du département de génie électrique et informatique. Un exemple serait un «end point» permettant au département d'avertir le «back end» de l'équipe lorsqu'un changement est effectué à l'horaire et de lui transmettre l'information concernant ce changement. Juste ce changement permettrait au serveur de l'équipe de répondre à une grande quantité d'utilisateurs sans compromettre les performances.

Ensuite, une recommandation nécessaire pour la continuité de l'application est de donner l'accès d'une machine virtuelle permanente à l'équipe. Cela donnerait la certitude à l'équipe que leurs services ne seront pas effacés du jour au lendemain lors d'une routine de ménage.

Quant à la troisième recommandation, celle-ci a été proposée par l'équipe professorale évaluant la présentation orale finale de l'équipe. Il s'agit de créer un groupe technique à l'université qui s'occuperait des outils informatiques créés par les étudiants. Cela permettrait que plusieurs projets bien développés puissent être offerts aux étudiants de GEGI et que les étudiants aient un groupe technique pour découvrir des langages qu'ils ne voient pas durant leur baccalauréat.

Finalement, l'équipe recommande que des employés du département informatique s'assoient avec les membres de l'équipe afin de discuter quels services disponibles sur GEL peuvent se retrouver dans l'application mobile. Par exemple, avertir les étudiants lorsqu'une nouvelle note est disponible ou afficher les messages disponibles sur la page d'accueil du site web.

\vfill

{%
\centering%
\includegraphics[width=0.3\textwidth]{Figures/logo}%
\par%
}

\vfill
