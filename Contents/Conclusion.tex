\section{Conclusion}
De toute évidence, le développement de l'application méritait une réflexion suivant une démarche responsable d'éthique, d'acceptabilité sociale et d'usage. De prime abord, il s'agit seulement d'une application affichant l'horaire personnel et envoyant des notifications de temps à autres. Nonobstant ces qualités, la démarche suivie est venue nuancer certains aspects importants du projet. Se penchant sur ceux les plus importants, l'équipe a déterminé que l'application devrait d'une part contrôler les notifications envoyées à ses utilisateurs afin de ne pas engendrer de dépendance envers le téléphone, tel que le suggèrent certaines études indiquées par la veille documentaire. \\

D'autre part, comme l'authentification à l'horaire donne indirectement accès à pratiquement tous les autres services de l'université, la sécurité derrière l'application se doit d'être robuste et sans failles. De bonnes mesures préventives permettront ainsi d'éviter les fuites d'informations et de préserver la confiance du public à l'égard de l'établissement. \\

Finalement, la veille documentaire a permis de mettre en lumière un aspect jusqu'alors resté dans l'oubli : l'inclusion. Les options d'accessibilité et les interfaces non genrées sont devenues omniprésentes au sein des systèmes d'exploitation, donc il reviendra à l'équipe d'en tirer parti afin de rendre l'application utilisable par le plus d'étudiants possible. \\

En somme, la démarche suivie a bel et bien permis à l'équipe de projet DEGEL d'ajuster le tir quant aux différents aspects du développement de l'application d'horaire. 

%TODO Maybe more bullshit at the end.