\section{Recommandations et conclusions}
Suivant la finalité d'un projet, il est nécessaire de réfléchir à certaines recommandations afin de prévoir la continuité du projet. La première recommandation, qui est probablement la plus importante, est la création d'\emph{endpoints} avec les services du département de GEGI. Un exemple serait un \emph{endpoint} de type \emph{callback} permettant au département d'avertir le \emph{backend} de l'équipe lorsqu'un changement est effectué à l'horaire et de lui transmettre l'information concernant ce changement. Ce simple changement permettrait au serveur de l'équipe de répondre à une plus grande quantité d'utilisateurs sans compromettre les performances générales.

Ensuite, une recommandation nécessaire pour la continuité de l'application est de donner l'accès d'une machine virtuelle permanente à l'équipe. Cela donnerait la certitude à l'équipe que leurs services ne seront pas effacés du jour au lendemain lors d'une routine de ménage.

Quant à la troisième recommandation, proposée par l'équipe professorale évaluant la présentation orale finale de l'équipe, consiste en la création d'un groupe technique à l'université qui s'occuperait des outils informatiques créés par les étudiants. Cela permettrait que plusieurs projets bien développés puissent être offerts aux étudiants de GEGI et que les étudiants aient un groupe technique pour découvrir des langages qu'ils ne voient pas durant leur baccalauréat.

Finalement, l'équipe recommande que des employés du département d'informatique prennent un moment pour discuter avec les membres de l'équipe des services disponibles sur GEL qui peuvent se retrouver dans l'application mobile. Par exemple, avertir les étudiants lorsqu'une nouvelle note est disponible ou afficher les messages disponibles sur la page d'accueil du site web.

{%
\centering%
$\star\star\star$
\par%
}

En guise de conclusion, l'équipe estime que le projet est une réussite, notamment par la publication sur l'Android Store et l'App Store de l'application DEGEL. Dorénavant, tous les étudiants (et les quelques étudiantes) du département de génie électrique et de génie informatique de l'Université de Sherbrooke pourront bénéficier du travail réalisé par l'équipe dans le cadre du cours de conception d'un système distribué pour consulter leur horaire sur mobile dans un format agréable, être informés des changements d'horaire de dernière minute et conserver une connexion au CAS durant toute une session.

L'équipe s'est montrée apte à surmonter les défis qui se sont présentés au fil des semaines, que ce soit des outils de gestion trop dispendieux, de la documentation de technologies parcellaire, des accès à \emph{Horarius} non obtenus ou bien des APP de modulation et de transmission RF interminables.

L'application des recommandations énoncées ci-haut ne peut qu'améliorer encore le produit livré et permettre à l'Université de Sherbrooke de faire bonne figure face à ses rivales montréalaises.



\vfill

{%
\centering%
\includegraphics[width=0.25\textwidth]{Figures/logo}%
\par%
}

\vfill
