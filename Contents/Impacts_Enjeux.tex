\section{Identification pondérée des impacts et des enjeux}
	\subsection{Restitution sous forme de tableaux}
		\subsubsection{Application mobile vs site web}
		Ajouter une application mobile à un service déjà existant apporte des bénéfices tels qu’une meilleure portabilité et une meilleure expérience utilisateur. Cette application mobile peut aussi ajouter de l’accessibilité aux services déjà en place. Toutefois, une application mobile implique une plus grande capture d’attention et ajoute de la maintenance à l’équipe informatique de l’Université de Sherbrooke. 
	
		\begin{sidewaystable}[p]
			\centering
			\caption{Bénéfices d'une application mobile}
			\label{tab.app+}
			\begin{tabular}{P{0.15\textwidth}P{0.15\textwidth}P{0.15\textwidth}P{0.15\textwidth}P{0.15\textwidth}P{0.15\textwidth}}
	\hline
	\bf Principe & \bf Usagers & \bf Conséquences & \bf Enjeux & \bf Défis & \bf Solutions \\
	\hline
	\hline
	Portabilité, usage sans internet
	& Étudiants, équipe professorale
	& Usage plus fréquent ?
	& Social : absentéisme
	& Comment réduire l'absentéisme ?
	& Comment les étudiants vérifient-ils l'horaire ?
	\\
	Expérience utilisateur, rapidité
	& Étudiants, équipe professorale
	& Usage plus agréable, inclusif
	& Éthique : info vs agréable + Éthique : simplicité vs inclusion
	& Manquer un examen la fin de semaine ? Interface genrée ?
	& Vérifier info actuelle + Consulter groupe LGBTQ2+
	\\
%	Accessibilité
%	& Usagers malvoyants
%	& Inclusion des personnes avec handicap
%	& Social
%	& Développement d'options d'accessibilité
%	& Tester l'application les yeux bandés
%	\\
	\hline
\end{tabular}
%		\end{sidewaystable}
		\bigskip\bigskip
%		\begin{sidewaystable}[p]
			\centering
			\caption{Inconvénients d'une application mobile}
			\label{tab.app-}
			\begin{tabular}{P{0.15\textwidth}P{0.15\textwidth}P{0.15\textwidth}P{0.15\textwidth}P{0.15\textwidth}P{0.15\textwidth}}
	\hline
	\bf Principe & \bf Usagers & \bf Conséquences & \bf Enjeux & \bf Défis & \bf Solutions \\
	\hline
	\hline
	Capture d'attention
	& Étudiants, équipe professorale
	& Rétention ?
	& Social : déconnexion sociale
	& Où est l'équilibre ?
	& Limiter les animations, information concise et précise
	\\
	\multirow{2}{0.15\textwidth}[-1em]
	{
		\begin{itemize}
			\item Temps de développement
			\item Intégration
			\item Maintenance
		\end{itemize}
	}
	& \tikzmark[xshift=-8pt,yshift=1ex]{x} Responsable de la sécurité
	& Possibles failles ?
	& Légal : fuites d'informations confidentielles
	& Comment sécuriser ?
	& Bonnes pratiques, consultation d'experts
	\\
	& \tikzmark[xshift=-8pt,yshift=-4ex]{y} Techniciens informatiques
	& Plus de travail, ont-ils les compétences ?
	& Économique : embauche, plus de temps
	& Minimiser les coûts de maintenance
	& Choix des langages, bonnes pratiques, documentation
	\\
	\hline
\end{tabular}

\drawbrace[brace mirrored, thick]{x}{y}
		\end{sidewaystable}
		
		\subsubsection{Notification vs aucune notification}
		L’ajout de notification à l’application mobile implique aussi une plus grande capture d’attention. Cependant, les notifications apportent aussi des bénéfices aux utilisateurs, car elles augmentent l’utilisation de l’application, ce qui la rend plus utile et elles avertissent les usagers lorsqu’il y a un changement à leur horaire. 
	
		\begin{sidewaystable}[p]
			\centering
			\caption{Bénéfices des notifications}
			\label{tab.notif+}
			\begin{tabular}{P{0.15\textwidth}P{0.15\textwidth}P{0.15\textwidth}P{0.15\textwidth}P{0.15\textwidth}P{0.15\textwidth}}
	\hline
	\bf Principe & \bf Usagers & \bf Conséquences & \bf Enjeux & \bf Défis & \bf Solutions \\
	\hline
	\hline
	Plus d'utilisation de l'application, application plus utile
	& Étudiants, équipe professorale
	& Éviter de manquer des événements importants
	& Social : pertinence des notifications
	& Assurer la pertinence des notifications ?, filtrer le contenu des notifications ?
	& Questionner les utilisateurs (contenu, fréquence), Regrouper les utilisateurs
	\\
	Avertissement de changement
	& Étudiants, équipe professorale, équipe de notification
	& Engagement à offrir un service fiable
	& Économique : maintient du service
	& Maintenir les serveurs, l'application ?
	& Découplage des fonctionnalités, couches interchangeables 
	\\
	
	\hline
\end{tabular}
%		\end{sidewaystable}
		\bigskip\bigskip
%		\begin{sidewaystable}[p]
			\centering
			\caption{Inconvénients des notifications}
			\label{tab.notif-}
			\begin{tabular}{P{0.15\textwidth}P{0.15\textwidth}P{0.15\textwidth}P{0.15\textwidth}P{0.15\textwidth}P{0.15\textwidth}}
	\hline
	\bf Principe & \bf Usagers & \bf Conséquences & \bf Enjeux & \bf Défis & \bf Solutions \\
	\hline
	\hline
	Capture d'attention
	& Étudiants, équipe professorale
	& Fréquence de consultation plus élevée
	& Social : dépendance aux notifications
	& Quelles sont les notifications importantes ?, Comment diminuer la fréquence ?
	& Donner le choix des notifications et permettre de désactiver les notifications
	\\
	\hline
\end{tabular}
		\end{sidewaystable}
		
		\subsubsection{Connexion longue durée vs connexion courte durée}
		La connexion longue durée d’une connexion apporte de la convivialité aux utilisateurs parce qu’ils n’ont pas à se connecter à chaque usage. Les jetons longues durées apportent plusieurs inconvénients aux responsables de la sécurité et du réseau sur le campus. En effet ils doivent être en mesure de révoqué un jeton à tout moment et d’assurer l’intégrité des informations. Pour ce qui est des responsables de l’horaire, ceux-ci doivent intégrer notre système dans le leur. En fin, une fuite d’information affecterait aussi les utilisateurs de l’application. 
	
		\begin{sidewaystable}[p]
			\centering
			\caption{Bénéfices d'une connexion longue durée}
			\label{tab.connexion+}
			\begin{tabular}{P{0.15\textwidth}P{0.15\textwidth}P{0.15\textwidth}P{0.15\textwidth}P{0.15\textwidth}P{0.15\textwidth}}
	\hline
	\bf Principe & \bf Usagers & \bf Conséquences & \bf Enjeux & \bf Défis & \bf Solutions \\
	\hline
	\hline
	Convivialité et efficacité
	& Étudiants, équipe professorale
	& Nul besoin de se reconnecter, Obtention de l'information plus rapidement
	& Social : Intégration
	& Technologie non adéquate
	& Développer système de notifications, discuter avec responsable de la sécurité
	\\
	\hline
\end{tabular}
%		\end{sidewaystable}
		\bigskip\bigskip
%		\begin{sidewaystable}[p]
			\centering
			\caption{Inconvénients d'une connexion longue durée}
			\begin{tabular}{P{0.15\textwidth}P{0.15\textwidth}P{0.15\textwidth}P{0.15\textwidth}P{0.15\textwidth}P{0.15\textwidth}}
	\hline
	\bf Principe & \bf Usagers & \bf Conséquences & \bf Enjeux & \bf Défis & \bf Solutions \\
	\hline
	\hline
	Révocation des permissions difficiles 
	& Responsable de la sécurité et du réseau informatique
	& Accès non autorisé
	& Légal : vie privée
	& Assurer la cohérence des permissions
	& Jetons de coure durée
	\\
	\multirow{2}{0.15\textwidth}[-4.5em]{Fuite possible d'informations}
	& \tikzmark[xshift=-8pt,yshift=-1ex]{x} \begin{itemize}
		\item Étudiants
		\item Équipe professorale
	\end{itemize}
	& Perte de contrôle des informations confidentielles
	& Légal : respect de la vie privée \hfill \tikzmark[xshift=-4pt,yshift=4ex]{m}
	& \multirow{2}{0.15\textwidth}[-3em]{Comment maintenir l'étanchéité du réseau informatique ?}
	&\multirow{2}{0.15\textwidth}[-3em]{Être capable de révoquer des jetons}
	\\
	& \tikzmark[xshift=-8pt,yshift=-6ex]{y} Responsable de la sécurité et du réseau informatique
	& Atteinte à la réputation 
	& Économique : taux d'inscription à l'université \hfill \tikzmark[xshift=-4pt,yshift=-1ex]{n}
	\\
	Intégration plus difficile
	& Responsable Horarius
	& \begin{itemize}
		\item Négociation mot de passe
		\item Délais de mise en place
	\end{itemize}
	& Social : échec du projet
	& Épreuve de diplomatie et de patience
	& \begin{itemize}
		\item Suivis serrés
		\item Avoir des solutions de rechange
	\end{itemize}
	\\
	\hline
\end{tabular}

\drawbrace[brace mirrored, thick]{x}{y}
\drawbrace[brace, thick]{m}{n}

			\label{tab.connexion-}
		\end{sidewaystable}
		
	\subsection{Priorité des enjeux et défis}
	L'établissement de priorités quant aux enjeux et défis de l'application permet d'ajuster le tir de la veille documentaire.
	
	\paragraph{Capture d'attention} Une application mobile capture nécessairement l'attention de son utilisateur. À long terme, ce comportement peut s'avérer nuisible s'il en résulte une déconnexion sociale des individus. En plus du contenu en soi, les notifications peuvent servir d'instrument de capture de l'attention et mener des utilisateurs à une addiction envers leur appareil. La veille documentaire porte ainsi sur les aspects de cet enjeu social, qui est pour le moins actuel chez les plus jeunes.
	
	\paragraph{Accessibilité} Les systèmes d'exploitation et les applications en général sont de plus en plus accessibles aux personnes présentant un handicap physique. La société occidentale étant dorénavant bien sensible à ces réalités de société, l'équipe souhaite inclure ces considérations dans la recherche afin d'établir les options les plus adaptées.
	
	\paragraph{Fuite d'information} L'application contient des données confidentielles par rapport au parcours académique des étudiants. La veille documentaire inclue donc une recherche par rapport aux enjeux légal et économique résultant d'une manipulation lâche des données. Ces efforts de recherche visent donc à minimiser les risques liés à l'utilisation de l'application mobile.
	
	
	
	
	
	
	
	
	
	
	
	
	
	