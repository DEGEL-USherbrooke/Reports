\section{Technique et gestion}
	\subsection{Suivi de l'avancement}
	\subsubsection{Backend}

	\paragraph{Authentification} Tout est terminé, hormis la déconnexion et la révocation de tokens.
	
	\paragraph{HTTPS} Le serveur est en HTTPS.
	
	\paragraph{Réglages} La base de données des réglages est créée, ainsi que son API.
	
	\paragraph{Droits d'accès à l'horaire} Il faudra passer par un iCal et demander directement à l'utilisateur de rentrer son lien iCal. \rWalley
	
	\paragraph{Base de données de l'horaire} Terminé en ce qui concerne la récupération de l'horaire, le storage et la différence entre chaque version de l'horaire.
	
\subsubsection{Mobile}

	\paragraph{Internationalisation} Après quelques petits bugs, cette fonctionnalité est en place. Il y a que les langages doivent être chargés de manière synchrone, avant le chargement des composants.
	
	\paragraph{Incohérences UI} Des incohérences au niveau de l'UI ont été corrigées avec succès. 
	
	\paragraph{Icône} L'application possède maintenant une icône unique et distinctive.
	
	\paragraph{Notifications} Il manque le backend par rapport au \emph{register} des notifications.
	
	\paragraph{CAS} L'interface de CAS est désormais intégrée dans une webview.
	
	\paragraph{Calendrier} La version 2 du calendrier est fonctionnelle.

\subsubsection{Gestion}

	\paragraph{GitLab} La licence GitLab a pris fin, donc l'équipe est sous la version gratuite et limitée de la plateforme. La licence pour établissements d'enseignement tarde étant donné le trop grand nombre d'institutions intéressées.
	
	\paragraph{JIRA} L'équipe manque de rigueur quant à la mise à jour des tâches, ce qui réduit en miettes le \emph{burndown chart}.
	
	\paragraph{Rapports} Toute l'équipe contribue aux rapports via GitLab sans avoir nécessairement besoin d'une distribution \LaTeX{} locale.
	
	


	\subsection{Écarts et révision de la planification}
	L'équipe fonctionne avec un barème de points (tableau~\ref{tab.bareme}). Le sprint~1 contient 53~points (104 à 141 heures). Plusieurs tâches reliées à Horarius ont dû être reportées, d'où la chute finale (figure~\ref{fig.burndowm}).
		
		
	\begin{minipage}{0.5\textwidth}
		\begin{table}[H]
			\centering
			\caption{Barème de points}
			\begin{tabular}{ll}
				\hline
				\bf Nombre de points & \bf Durée (heures) \\
				\hline
				\hline
				1 & 0,5 - 1 	\\
				2 & 3 - 4 	\\
				3 & 5 - 8		\\
				4 & 9 - 12	\\
				5 & 13+ 	\\
				\hline
			\end{tabular}
			\label{tab.bareme}
		\end{table}		
	\end{minipage}%
	%
	\begin{minipage}{0.5\textwidth}
		\begin{figure}[H]
			\centering
			\includegraphics[width=0.9\textwidth]{Figures/burndownChart}
			\caption{Burndown Chart du sprint 1}
			\label{fig.burndowm}
		\end{figure}		
	\end{minipage}


	

	\subsection{Prévisions pour la prochaine itération}
	\subsubsection{Backend}
	\paragraph{Horaire} Gestion complète de l'horaire (récupération sur Horarius, différence entre fichiers iCal).

	\paragraph{Notifications} Construire un pont avec le système de notifications.
	
	\paragraph{Déconnexion} Développer la fonctionnalité de déconnexion et de révocation de tokens.
	
\subsubsection{Mobile}
	\paragraph{Horaire} Traiter le format de données iCal.
	
	\paragraph{Notifications} Gérer les tokens expo. Intégration avec l'équipe de notifications. Gestion de l'envoi des notifications.
	
	\paragraph{Accessibilité} Rendre l'application accessible aux personnes malvoyantes.
	
	\paragraph{Bouton Today} Ajout d'un bouton retournant à la journée courante.
	
	\paragraph{Varia} Corrections de bugs variés.
	


	\subsection{Risques importants}
	\section{Problèmes rencontrés}
	\subsection{Backend}
	\paragraph{Horarius} L'accès au JSON d'horarius s'avère plus complexe que prévu. La version iCal n'est pas présentée de la même manière, même si convertie en JSON, ce qui nécessiterait davantage de manipulation de données.
	
	\subsection{Mobile}
	L'offre de frameworks de calendriers en React Native avoisine la nullité. Cependant, l'équipe ne veut pas faire la conception calendrier à partir de rien, car c'est un défi énorme. Il y a beaucoup de \emph{edge cases} à penser et il est très difficile de faire du beau UI sans un designer très expérimenté. Nous tenterons jusqu'au bout d'utiliser les \emph{plugins} qui existent et de les étendres avant de faire notre propre implémentation.
	
	\subsection{Gestion}
	L'équipe aura avantage à instaurer des normes à suivre sur Git. Se rapprochant de la fin du baccalauréat, chaque membre de l'équipe possède sa propre expérience en entreprise de l'outil et sa propre définition de \og bonnes pratiques \fg, ce qui mène parfois à de petits débats.