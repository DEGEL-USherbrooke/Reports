\section{Acquisition de connaissance}

	\subsection{Connaissances théoriques}
		\subsubsection{Plateformes}
		Au Québec, chaque campus universitaire possède son propre interface d'horaire académique. En guise d'exemples, l'Université de Montréal offre une plateforme partagée entre tous les niveaux d'études et permet d'afficher des soutenances de thèses et des cours d'été disponibles; l'Université du Québec en Outaouais et l'Université Laval offrent chacune un engin de recherche de cours.
		
		\subsubsection{Addiction et nomophobie}
		
		{\bfseries The relationship between cell phone use, academic performance, anxiety, and Satisfaction with Life in college students}	
		
		Cette étude américaine de Kent State University porte sur l'effet de l'utilisation des téléphones portables sur les résultats académiques, le niveau de satisfaction et l‘anxiété de ses étudiants. Selon les résultats obtenus, une l'utilisation des téléphones mobiles tend à diminuer les deux premiers facteurs et à augmenter le dernier de façon significative. \\
		
		{\bfseries The psychological reason you can't stop checking your phone}
		
		Cet article traite des sources de l'addiction aux téléphones. Plus spécifiquement, il traite du fait que l'addiction provient plus de l'anticipation d'une notification que de l'information qu'elle porte. C'est cette anticipation qui crée une augmentation de la dopamine associée au plaisir et qui conduit à l'addiction. \\
		
		{\bfseries Mésusage du téléphone mobile et nomophobie chez les étudiants}
		
		Cette article met en lumière la relation entre la dépendance aux mobiles et les troubles du sommeil. Cette dépendance peut avoir des conséquences sur les performances académiques des étudiants.
				
		\subsubsection{Sécurité}
		{\bfseries Privacy and security in mobile health apps: a review and recommendations}
		
		Cette recherche porte sur la sécurité des applications mobiles pour la santé en Europe et aux USA. Comme ces applications travaillent avec des données très sensibles, il est primordial qu'elles soient sécuritaires. La recherche fait d'abord le tour des lois en place (EU Data Protection Directive, Federal Trade Commission Act section 5, etc.) et avance plusieurs recommendation afin d'assurer une sécurité maximale des applications.
		
		%TODO Ajouter sur H. Maddison (xavier)
		
		\subsubsection{Inclusion}
		{\bfseries A Survey on the Use of Mobile Applications for People Who Are Visually Impaired}
		
		%TODO traduire en FR
		\begin{itemize}
			\item A total of 259 participants with visual impairments completed an online survey. 
			\item The participants rated special apps as useful (95.4\%) and accessible (91.1\%) tools for individuals with visual impairments.
			\item Results show that persons with visual impairments frequently use apps specifically designed for them to accomplish daily activities.
			\item Furthermore, this population is satisfied with mobile apps and would like to see improvements and new apps.
			\item Developers of apps for individuals with visual impairments need to refine and test the existing apps. 
		\end{itemize}
		
		\subsubsection{Éducation}
		{\bfseries Mobile applications in university education: the case of Kenya}
		
		% TODO Traduire en FR
		\begin{itemize}
			\item For any university that is innovative and continuously exploring new strategies for education, ignoring the potential of mobile apps is generally detrimental to progress, and it would imply refusal to adapt in a continuously changing world. 
			\item also administration and other activities could beneft from students behaviors and needs in relation to mobile technologies and apps.
		\end{itemize}
	
	\subsection{Connaissances de terrain}
	La cible visée par l'acquisition de connaissances de terrain est les usagers potentiels de l'application. Celle-ci s'adresse spécifiquement aux étudiants, professeurs et chargés de cours du département de génie électrique et génie informatique puisque la gestion d'horaire ne concerne que ce département. \\
	
	Dans le cadre de cette acquisition de connaissances, les thématiques suivantes seront abordées avec la cible précédemment mentionnée :

	\begin{enumerate}
		% Telephone
		\item Quelle est votre relation avec votre téléphone ?
		\begin{itemize}
			\item Quelle est votre fréquence d'utilisation de votre téléphone ?
			\item Consultez-vous vos sites internet favoris sur mobile ?
			\item Aimez-vous recevoir des notifications ? Si oui, de quels genres ?
		\end{itemize}
		% UNI
		\item Quelle est votre relation avec les services mobiles universitaires existants ?
		\begin{itemize}
			\item En utilisez-vous ?
			\item Quel est votre niveau de satisfaction ?
			\item En voudriez-vous plus ?
		\end{itemize}
		% Horaire
		\item Comment consultez-vous votre horaire actuellement ?
		\begin{itemize}
			\item Sur quelle plateforme ?
			\item À quelle fréquence ?
			\item À quel moment de la journée ?
			\item Comment trouvez-vous la méthode d'authentification au service ?
		\end{itemize}
	\end{enumerate}
	
	\subsection{Enrichissement}
	% en quoi la veille a enrichi vos enjeux et défis et les à éclairer ? Que retenez-vous ?
	
		\subsubsection{Développement des notifications}

		Suivant la recherche documentaire, on en retire que l'utilisations du téléphone peut mener à une hausse de l'anxiété, à une diminution de la satisfaction sa propre vie et des résultats scolaires. Il s'agit donc d'enjeux sérieux dont il faudra fortement tenir compte lors du développement de l'application, notamment à cause que l'utilisation du téléphone chez les milléniaux est presque rendu un problème de santé publique. \\
		
		Anticiper des notifications est associé à la sécrétion de dopamine, l'hormone de l'addiction. Il faudra donc que l'utilisateur soit en mesure de régler la \emph{granularité} des notifications reçues, pas rapport au son, à la vibration et au texte en soi. Il sera donc avantageux de limiter les notifications au contenu véritablement essentiel, par exemple en regroupant les changements consécutifs au même évènement et en ignorant les changements apportés aux évènements du passé ou du futur lointain. \\
			
		L'utilisation du téléphone à certaines heures peut perturber le sommeil. Une configuration du téléphone doit donc permettre de détecter les périodes de sommeil afin de ne pas déranger le propriétaire du téléphone. \\
		
		En somme, la veille documentaire nous a permis de réaliser qu'il sera avantageux de réduire les effets négatifs des notifications.
	
		\subsubsection{Sécurité}
		Des lois strictes existent au Canada de nos jours quant à la manipulation de données personnelles. Il en va de même aux États-Unis. Il sera à notre avantage d'en prendre connaissance et de ne pas négliger cet aspect de sécurité afin de nous éviter des problèmes légaux. \\
		
		Notre enrichissement quant aux bonnes pratiques nous amène donc à suivre les recommandations qui s'appliquent au contexte de notre application de gestion d'horaire académique. De plus en plus fréquemment, les fuites d'informations personnelles font couler beaucoup d'encre dans les journaux. Afin de ne pas nuire à la réputation de l'Université, ce qui pourrait avoir un impact monétaire en diminuant le taux d'inscription de l'établissement. Par ailleurs, le fait que la technologie soit de plus en plus présente dans nos vies rend cet aspect plus qu'important puisque chaque fuite a le potentiel de faire plus de dommage que la précédente. \\
		
		En bref, la sécurité sera au c\oe ur de notre application, non un aspect à part. 
		
		
	
	
	
	
	
	