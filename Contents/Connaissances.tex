\section{Acquisition de connaissance}

	\subsection{Connaissances théoriques}
	\subsection{Connaissances de terrain}
	La cible visée par l'acquisition de connaissances de terrain est les usagers potentiels de l'application. Celle-ci s'adresse spécifiquement aux étudiants, professeurs et chargés de cours du département de génie électrique et génie informatique puisque la gestion d'horaire est particulière à ce département. \\
	
	Dans le cadre de cette acquisition de connaissances, les thématiques suivantes seront abordées avec la cible précédemment mentionnée :

	\begin{enumerate}
		% Telephone
		\item Quelle est votre relation avec votre téléphone ?
		\begin{itemize}
			\item Quelle est votre fréquence d'utilisation de votre téléphone ?
			\item Consultez-vous vos sites internet favoris sur mobile ?
			\item Aimez-vous recevoir des notifications ? De quels genres ?
		\end{itemize}
		% UNI
		\item Quelle est votre relation avec les services mobiles universitaires existants ?
		\begin{itemize}
			\item En utilisez-vous ?
			\item Quel est votre niveau de satisfaction ?
			\item En voudriez-vous plus ?
		\end{itemize}
		% Horaire
		\item Comment consommez-vous votre horaire actuellement ?
		\begin{itemize}
			\item Sur quelle plateforme ?
			\item À quelle fréquence ?
			\item À quel moment de la journée ?
			\item Comment trouvez-vous l'authentification au service ?
		\end{itemize}
	\end{enumerate}
	
	\subsection{Enrichissement}
	% en quoi la veille a enrichi vos enjeux et défis et les à éclairer ? Que retenez-vous ?
	
	