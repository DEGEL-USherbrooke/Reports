\section{Acquisition de connaissance}

	\subsection{Connaissances théoriques}
		\subsubsection{Plateformes}
		Au Québec, chaque campus universitaire possède sa propre interface d'horaire académique. En guise d'exemples, l'Université de Montréal offre une plateforme partagée entre tous les niveaux d'études et permet d'afficher des soutenances de thèses et des cours d'été disponibles; l'Université du Québec en Outaouais et l'Université Laval offrent chacune un engin de recherche de cours.
		
		\subsubsection{Addiction et nomophobie}
		
		{\bfseries \cite{relationship} The relationship between cell phone use, academic performance, anxiety, and Satisfaction with Life in college students}	
		
		Cette étude américaine de Kent State University porte sur l'effet de l'utilisation des téléphones portables sur les résultats académiques, le niveau de satisfaction et l'anxiété de ses étudiants. Selon les résultats obtenus, une l'utilisation des téléphones mobiles tend à diminuer les deux premiers facteurs et à augmenter le dernier de façon significative. \\
		
		{\bfseries \cite{cannotStop} The psychological reason you can't stop checking your phone}
		
		Cet article traite des sources de l'addiction aux téléphones. Plus spécifiquement, il traite du fait que l'addiction provient plus de l'anticipation d'une notification que de l'information qu'elle porte. C'est cette anticipation qui crée une augmentation de la dopamine associée au plaisir et qui conduit à l'addiction. \\
		
		{\bfseries \cite{mesusage} Mésusage du téléphone mobile et nomophobie chez les étudiants}
		
		Cette article met en lumière la relation entre la dépendance aux mobiles et les troubles du sommeil. Cette dépendance peut avoir des conséquences sur les performances académiques des étudiants.
				
		\subsubsection{Sécurité}
		{\bfseries \cite{privacy} Privacy and security in mobile health apps: a review and recommendations}
		
		Cette recherche porte sur la sécurité des applications mobiles pour la santé en Europe et aux USA. Comme ces applications travaillent avec des données très sensibles, il est primordial qu'elles soient sécuritaires. La recherche fait d'abord le tour des lois en place (EU Data Protection Directive, Federal Trade Commission Act section 5, etc.) et avance plusieurs recommandation afin d'assurer une sécurité maximale des applications.
		
		{\bfseries \cite{dataBreach} The 17~biggest data breaches of the 21st century}
		Cet article revient sur les plus grosses fuites de données personnelles à ce jour. Pour chaque fuite, un estimé des coûts est fait et le tout se chiffre en milliards de dollars. L'article observe également l'accélération du nombre de fuites importantes au cours des dernières années.
		
		
		\subsubsection{Inclusion}
		{\bfseries \cite{visuallyImpaired} A Survey on the Use of Mobile Applications for People Who Are Visually Impaired}
		
		Cette étude démontre que la population des personnes avec des troubles de vision utilise fortement leur téléphone intelligent. Cependant, ils dépendent énormément des fonctionnalités d'accessibilité que les développeurs peuvent intégrer à leur application. Il est aussi démontré que les gens avec des troubles de vision vont utiliser beaucoup plus des applications avec ces fonctionnalités. \\
		
		{\bfseries \cite{genderNeutral} Designing the Gender-Neutral User Experience}
		
		Cette thèse étudie le biais de genre dans le design des interfaces utilisateurs notamment dans le choix des couleurs, des formes et dans le choix du langage. Elle tente également d'établir des critères et recommandations pour les interfaces neutres. Elle contient également une étude qui conclut que les interfaces neutres sont appréciées de façon similaire par tous les genres et que c'est la tranche d'âge qui détermine le plus le niveau d'appréciation de ces interfaces.
		
		\subsubsection{Éducation}
		{\bfseries \cite{kenya} Mobile applications in university education: the case of Kenya}
		
		Cette étude brosse le portrait de l'utilisation des téléphones intelligents par les étudiants universitaires kényens et du potentiel éducatif d'avoir des applications mobiles dans les universités. Selon l'étude, ignorer le potentiel des applications mobile est nuisible pour toute université dite innovante. Également, la demande la plus fréquente des étudiants était d'avoir une application pour consulter leur horaire, les nouvelles de l'université et recevoir/remettre des devoirs.
	
	\subsection{Connaissances de terrain}
	La cible visée par l'acquisition de connaissances de terrain est les usagers potentiels de l'application. Celle-ci s'adresse spécifiquement aux étudiants, professeurs et chargés de cours du département de génie électrique et génie informatique puisque la gestion d'horaire ne concerne que ce département. Dans le cadre de cette acquisition de connaissances, les thématiques suivantes seront abordées avec la cible précédemment mentionnée :

	\begin{enumerate}
		% Telephone
		\item Quelle est votre relation avec votre téléphone ?
		\begin{itemize}
			\item Quelle est votre fréquence d'utilisation de votre téléphone ?
			\item Consultez-vous vos sites internet favoris sur mobile ?
			\item Aimez-vous recevoir des notifications ? Si oui, de quels genres ?
		\end{itemize}
		% UNI
		\item Quelle est votre relation avec les services mobiles universitaires existants ?
		\begin{itemize}
			\item En utilisez-vous ?
			\item Quel est votre niveau de satisfaction ?
			\item En voudriez-vous plus ?
		\end{itemize}
		% Horaire
		\item Comment consultez-vous votre horaire actuellement ?
		\begin{itemize}
			\item Sur quelle plateforme ?
			\item À quelle fréquence ?
			\item À quel moment de la journée ?
			\item Comment trouvez-vous la méthode d'authentification au service ?
		\end{itemize}
	\end{enumerate}
	
	\subsection{Enrichissement}
	% en quoi la veille a enrichi vos enjeux et défis et les à éclairer ? Que retenez-vous ?
	
		\subsubsection{Développement des notifications}

		Suivant la recherche documentaire, on en retire que l'utilisations du téléphone peut mener à une hausse de l'anxiété, à une diminution de la satisfaction sa propre vie et des résultats scolaires. Il s'agit donc d'enjeux sérieux dont il faudra fortement tenir compte lors du développement de l'application, notamment puisque l'utilisation du téléphone chez les milléniaux est presque rendu un problème de santé publique. \\
		
		Anticiper des notifications est associé à la sécrétion de dopamine, l'hormone de l'addiction. Il faudra donc que l'utilisateur soit en mesure de régler la \emph{granularité} des notifications reçues, pas rapport au son, à la vibration et au texte en soi. Il sera donc avantageux de limiter les notifications au contenu véritablement essentiel, par exemple en regroupant les changements consécutifs au même évènement et en ignorant les changements apportés aux évènements du passé ou du futur lointain. \\
			
		L'utilisation du téléphone à certaines heures peut perturber le sommeil. Une configuration du téléphone doit donc permettre de détecter les périodes de sommeil afin de ne pas déranger le propriétaire du téléphone. \\
		
		En somme, la veille documentaire nous a permis de réaliser qu'il sera nécessaire de réduire les effets négatifs des notifications.
	
		\subsubsection{Sécurité}
		Des lois strictes existent au Canada de nos jours quant à la manipulation de données personnelles. Il en va de même aux États-Unis et dans l'UE. Il sera à notre avantage d'en prendre connaissance et de ne pas négliger cet aspect de sécurité afin de nous éviter des problèmes légaux. \\
		
		Notre enrichissement quant aux bonnes pratiques nous amène donc à suivre les recommandations qui s'appliquent au contexte de notre application de gestion d'horaire académique. De plus en plus fréquemment, les fuites d'informations personnelles font couler beaucoup d'encre dans les journaux. Afin de ne pas nuire à la réputation de l'Université, ce qui pourrait avoir un impact monétaire en diminuant le taux d'inscription de l'établissement. Par ailleurs, le fait que la technologie soit de plus en plus présente dans nos vies rend cet aspect plus qu'important puisque chaque fuite a le potentiel de faire plus de dommage que la précédente. \\
		
		En bref, la sécurité sera au c\oe ur de notre application, non un aspect à part. 
		
		\subsection{Inclusion}
		L'inclusion de tous les utilisateurs est un sujet difficile, mais essentiel à considérer lors de la conception. Tout d'abord, il faut éviter de créer une interface utilisateur genrée. La veille a permis de prendre mesure de l'ampleur de ce que cela veut dire. Du choix des couleurs aux formes utilisées en passant par le choix des mots employés. Tout doit être réfléchi afin d'offrir une expérience inclusive. Un autre point clé à retenir est que lorsqu'il est n'est pas absolument essentiel d'avoir ou d'utiliser le genre, l'application devrait tout simplement s'abstenir de le faire.
		
		Avoir une interface inclusive, c'est également s'adapter aux personnes présentant des handicaps. La veille nous a permis de prendre conscience de cet enjeux qui n'était pas ressorti pendant l'identification des impacts et enjeux. Il nous semble que la communauté de développement mobile est moins sensibilisé à cet enjeux que celle de développement d'applications pour ordinateur. Les outils existent pour développer des applications accessibles à cette population, mais les magasins d'applications ne l'exigent pas encore.
		
		Bref, développer une interface inclusive n'est pas une tâche aisée, mais essentielle en 2018.
		
		\subsection{Éducation}
		%TODO: say more bullshit here
		Finalement, il nous semblait important de revenir sur les multiples impacts positifs importants que peuvent avoir les applications mobiles dans un cadre éducatif comme celui de l'université. Même si nous étudions déjà convaincus de ce fait, la veille nous a permis de trouver des sources externes appuyant notre intuition. Nous  sommes donc confiants du bien fait qu'apportera notre application à l'université.
		
		
	
	
	
	
	
	